\documentclass{seneca_final}
\usepackage{blindtext}

\usepackage{array}
\newcolumntype{R}{>{\raggedleft\arraybackslash}b{4cm}}
%\usepackage[left=1.25in, right=1.25in, top=1.25in, bottom=1.5in]{geometry} %Sets
%Write the name of your class and the term below
%Make sure to keep the double curly braces
\CourseTerm{{OPS105 Final Exam}}{{Winter 2016}}
\CollegeDept{{Seneca College}}{{School of ICT}}


\begin{document}

\thispagestyle{empty}
\begin{center}
\noindent
\MakeUppercase{\textbf{Seneca College of Arts and Technology}}
\vspace{0.5cm}

\noindent
\MakeUppercase{\textbf{School of Information and Communication Technology - SY}}
\vspace{0.5cm}

\noindent
\MakeUppercase{\textbf{Final Examination - Version A}}
\vspace{0.5cm}
\end{center}

%\renewcommand{\arraystretch}{1.5}
\begin{tabular}{ p{3.5cm} p{7.0cm} R }
 \MakeUppercase{SEMESTER} & \MakeUppercase{SUBJECT NAME}  & \MakeUppercase{SUBJECT CODE} \\ \hline
  & & \\
  Winter 2016 & Operating Systems: Practices & OPS105 \\ \hline
\end{tabular}
\vspace{0.5cm}

\begin{center}
\MakeUppercase{name:}\underline{\hspace{8.5cm}} \\
\vspace{0.25cm}
\MakeUppercase{student number:}\underline{\hspace{6cm}}\\
\vspace{0.25cm}
\MakeUppercase{section:}\underline{\hspace{8.2cm}}
\vspace{0.5cm}
\end{center}


\begin{tabular}{ p{3.5cm} p{7.0cm} p{0.5cm} p{1.5cm}}
\MakeUppercase{date:} & Monday, April 18, 2015  & & \\
  & & & \\
\MakeUppercase{time allowed:} & 2 hours& & \\
  & & & \\
\MakeUppercase{questions:} & & & \\
\MakeUppercase{part a} & Multiple Choice & 10 & \MakeUppercase{marks}\\
\MakeUppercase{part b} & Multiple Choice & 10 & \MakeUppercase{marks}\\
\MakeUppercase{part c} & Multiple Choice & 10 & \MakeUppercase{marks}\\
\MakeUppercase{} & & & \\
& \MakeUppercase{total marks:} & 30 &\\
& & & \\
\MakeUppercase{Professor(s):} & Marcel Jar & & \\
\MakeUppercase{} & & & \\
\MakeUppercase{} & & & \\
\end{tabular}

\noindent
\textbf{\underline{\MakeUppercase{special instructions:}}}

\begin{enumerate}
   \item This is a closed book exam. No reference sheets or aids material of any are allowed.
   \item Write your answers in the spaces provided
\end{enumerate}

\noindent
This exam includes a cover page, plus 7 pages of questions.
\vspace{0.15cm}
\hrule

\vspace{0.5cm}
\noindent
\MakeUppercase{Seneca's academic honesty policy:}

\noindent
As a Seneca student, you must conduct yourself in an honest and trustworthy manner in all aspects of your academic career. A dishonest attempt to obtain an academic advantage is considered an offense, and will not be tolerated by the College.

\vspace{0.5cm}
\noindent
\MakeUppercase{Approved by:}

\noindent
Mary Lynn Manton, Chair, School of ICT

\newpage

   \Part{Multiple Choice}

   \Question{Here goes some random text that should be part of the question. I will keep on writing for a while just to ensure that we have multiple lines.}
   \begin{multiple_choice}
      \item Again I will write a lot jsut to guarantee that we will haev more than jsut one line of text.
      \item test2
      \item test2
      \item test2
      \item test2
   \end{multiple_choice}

   \Question{Here goes some random text that should be part of the question. I will keep on writing for a while just to ensure that we have multiple lines.}
   \begin{true_false}
      \item Again I will write a lot jsut to guarantee that we will haev more than jsut one line of text.
      \item test2
      \item test2
      \item test2
      \item test2
   \end{true_false}

   \Question{Here goes some random text that should be part of the question. I will keep on writing for a while just to ensure that we have multiple lines.}
   \Subquestion{Here goes some random text that should be part of the question. I will keep on writing for a while just to ensure that we have multiple lines.}
   \Subquestion{Here goes some random text that should be part of the question. I will keep on writing for a while just to ensure that we have multiple lines.}


   \Question{Here goes some random text that should be part of the question. I will keep on writing for a while just to ensure that we have multiple lines.}
   \begin{match_a}
      \item Again I will write a lot jsut to guarantee that we will haev more than jsut one line of text.
      \item test2
      \item test2
      \item test2
      \item test2
   \end{match_a}
   \begin{match_b}
      \item Again I will write a lot jsut to guarantee that we will haev more than jsut one line of text.
      \item test2
      \item test2
      \item test2
      \item test2
   \end{match_b}


\Lines{4}{1cm}

   \blinddocument

\end{document}
